\section{Główne osiągnięcia pracy}
Głównym celem pracy było utworzenie \foreignquote{english}{proof of concept} oprogramowania analizującego ruch na systemie plików w celu wykrycia ataku ransomware. Rozwiązanie spełniło wszystkie wymagania jakościowe i funkcjonalne zdefiniowane w sekcji \hyperref[sec:wymagania]{Specyfikacja wymagań funkcjonalnych~i~niefunkcjonalnych}. Udało się utworzyć oprogramowanie pozyskujące informacje o operacjach w celu dalszej analizy pod kątem ataku oprogramowaniem złośliwym. Wykorzystane zostały znane w ekosystemie Linuksowym technologie, które bez wątpienia każdy administrator będzie mógł w efektywny i wygodny sposób dostosować do własnych potrzeb.
\newline
Wyniki testów wyjawiły brak dopracowanych, realistycznych wymagań jakie powinny spełniać analizowane środowiska oraz to w jakim toku eksperymenty powinny być przeprowadzane w przyszłości. Jasne stało się, że założenie o ataku objawiającym się w czasie mniejszym niż doba było błędne. Kluczeowym przy analizie zachowań wirusów ransomware jest wykorzystanie rozwiązania w warunkach symulujące praktyczne środowiska produkcyjne w okresie conajmniej kilku tygodni. Mimo to, udało się wyciągnąć inne interesujące wnioski na temat skuteczności i potrzeby dywersyfikacji wykorzystywanych metod. 
\section{Ograniczenia proponowanej metody}
Największym ograniczeniem zaproponowanej metody jest zależność od zasad audytowania systemu. Mimo iż możliwe jest utworzenie zupełnie odrębnego rozwiązania zbierającego informacje o operacjach na systemie, musiałoby to się wiązać z potrzebą dystrybuowania specjalnej wersji jądra systemowego lub wprowadzenia proponowanych zmian na głównym repozytorium projektu Linux. Ponad to Linux Auditing Framework również sam z siebie wpływa na wydajność systemu. W związku z tym na administratorze niestety spoczywa obowiązek dostosowania zasad audytu pod względem zużycia zasobów.
\newpage
Ograniczająca także jest zależność rozwiązania operającego się na wyliczaniu różnicy w polu entropii na plikach w zależności od rozpatrywanego formatu. W mojej implementacji opierałem się na danych zebranych z pracy \foreignquote{english}{Differential area analysis for ransomware attack detection within mixed file datasets}~\cite{davies_differential_2021} w której ilość próbek, na podstawie której wyznaczono graniczne wartości różnicy pól, była stosunkowo niewielka. Należałoby zrobić serię badań dla większej ilości formatów z większą ilością próbek. Wymaga to pozyskiwania danych od klientów lub przy pomocy kontrolowanych ataków na wyizolowanych wariantach wirusów. To samo tyczy się metody podobieństwa cosinusowego z dodatkową wadą wymogu analizy tekstowej kodu asemblerowego plików posiadających potencjalnie tysiące linijek.

\section{Propozycje dalszego rozwoju i doskonalenia systemu}

Aby można było rozwinąć koncept przedstawionego oprogramowania w godny uwagi produkt, należałoby zaimplementować możliwość utworzenia harmonogramu zadań. Zadaniami w tym modelu byłyby skany poszczególnymi metodami lub dowolnej ich kombinacją. Z punktu widzenia administracji systemami, dobrym dodatkiem byłby także interfejs graficzny i klient CLI ułatwiający automatyzację. Należałoby także, tak jak wymieniłem w poprzedniej sekcji, dokonać badań na większej ilości zaszyfrowanych plików o szerokiej gamie rozszerzeń oraz sformułować i zaimplementować inne rodzaje algorytmów analizujących ruch na systemie plików.