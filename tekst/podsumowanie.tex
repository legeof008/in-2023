\section{Główne osiągnięcia pracy}
Głównym celem pracy było utworzenie \foreignquote{english}{proof of concept} oprogramowania analizującego ruch na systemie plików w celu wykrycia ataku ransomware. Ten cel został w dużej mierze osiągnięty. Rozwiązanie spełniło wszystkie wymagania jakościowe i funkcjonalne zdefiniowane w sekcji \hyperref[sec:wymagania]{Specyfikacja wymagań funkcjonalnych~i~niefunkcjonalnych}. Udało się przede wszystkim utworzyć oprogramowanie będące silnikiem napędzającym dalsze metody analizy pod kątem ataku oprogramowaniem złośliwym. Wykorzystane zostały znane w branży administracji systemami metody, które bez wątpienia każdy administrator będzie w stanie w efektywny i wygodny sposób dostosować do własnych potrzeb.
\newline
Przeprowadzenie praktycznych testów także było pomocne w dopracowaniu zakresu rzeczywistych wymagań jakie powinny spełniać analizowane środowiska oraz w jakim toku tego typu eksperymenty powinny być przeprowadzane. Wyniki testów wskazują na to, że założenie o ataku objawiającym się w mniej niż dobę było błędne. Kluczem w analizie zachowań wirusów ransomware jest użycie rozwiązania w warunkach praktycznych w okresie conajmniej kilku tygodni. Mimo tego niedopatrzenia udało się wyciągnąć inne interesujące wnioski na temat skuteczności i potrzeby dywersyfikacji wykorzystywanych metod. 
\section{Ograniczenia proponowanej metody}
Największym ograniczeniem zaproponowanej metody jest zależność od zasad audytowania systemu. Możliwe byłoby utworzenie zupełnie odrębnego rozwiązania ale musiałoby to się wiązać z potrzebą dystrybuowania specjalnej wersji jądra systemowego lub akceptacji zmian na głównym repozytorium projektu Linux. Dodatkowo Linux Auditing Framework również sam z siebie wpływa na wydajność systemu. W związku z tym na administratorze niestety spoczywa obowiązek dostosowania zasad audytu pod względem zużycia zasobów, również biorąc pod uwagę dodatkowe rozwiązania, które się na nich opierają.
\newpage
Ograniczająca także jest zależność rozwiązania operającego się na wyliczaniu różnicy w polu entropii na danych w zależności od rozpatrywanego formatu. W mojej implementacji opierałem się na danych zebranych z pracy \foreignquote{english}{Differential area analysis for ransomware attack detection within mixed file datasets}~\cite{davies_differential_2021} jednak ilość próbek, na podstawie której wyznaczono graniczne wartości różnicy pól niewielkie. Należałoby zrobić serię badań dla większej ilości formatów z większą ilością próbek. To samo tyczy się metody podobieństwa cosinusowego z dodatkową wadą wymogu analizy tekstowej kodu asemblerowego plików posiadających potencjalnie tysiące linijek.

\section{Propozycje dalszego rozwoju i doskonalenia systemu}

Aby można było rozwinąć koncept przedstawionego oprogramowania w godny uwagi produkt, należaoby zaimplementować możliwość utworzenia harmonogramu zadań. Zadaniami w tym modelu byłyby skany poszczególnymi metodami lub dowolnej ich kombinacją. Z punktu widzenia administracji systemami, dobrym dodatkiem byłby także interfejs graficzny i klient CLI ułatwiający automatyzację. Należałoby także, tak jak wymieniłem w poprzedniej sekcji, dokonać badań na większej ilości zaszyfrowanych plików o szerokiej gamie rozszerzeń, a także sformułować i zaimplementować inne rodzaje algorytmów analizujących ruch na systemie plików.